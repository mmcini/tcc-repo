Soils play a key role in human activities; they are the basis of food
production, sustain the majority of human infrastructure and are the cradle of
most of the vegetation above the continental crust. However, soils are very
heterogeneous and vary not only along the landscape, but also with depth. Such
variability has direct impact in the possible uses of each soil. Currently, soil
studies use mainly visual and mechanical methods to characterize soils in-situ,
later complemented by laboratory analyses. The objective of this study was to
test new technologies –proximal sensors– to improve the methodologies of soil
profile characterization and soil genesis studies. Three soil profiles with
buried genetic horizons were studied; i.e., soils in which a mantle of sediments
was deposited above a preexisting soil. Soil profiles were located in Lavras,
Minas Gerais, Brazil, and were classified as Haplic Organosol (OX), Haplic
Cambisol (CX) and Haplic Gleysol (GX). Profiles were analyzed in the field and
sampled in a grid with 8 lines and 5 columns (15 x 15 cm), summing 40 samples
per profile. Besides physical and chemical analyses, soils were analyzed by
portable X-ray fluorescence spectrometry (pXRF) and visible-infrared diffuse
reflectance spectroscopy (Vis-NIR). Data from pXRF were spatialized along
profiles to assess the spatial distribution of elemental contents within and
between the horizons. There was remarkable variation of chemical and physical
characteristics between buried genetic horizons. Spectral data from Vis-NIR
identified different characteristics within the histic horizon of OX as a
function of depth, and highlighted differences in the organic matter content in
the buried A horizon of CX. The spatial distribution of elemental contents
obtained from pXRF showed discontinuities in the chemical composition of the
three profiles close to 110 cm of depth; an indication of chemical differences
in the parent material. Both sensors detected higher contents of iron oxides in
the superficial deposited mantle of OX and CX, suggesting that sediments were
originated from Latosols or Argisols derived from granite-gneiss and gabbro,
common soil class and parent materials in the region. Proximal sensors made
clear that the three soils presented sediment deposition from higher surrounding
areas and, hence, presented parent material discontinuity. Sensors were capable
of detecting variations within horizons and quantifying key properties of soils
that differentiate horizons from each other according to their genesis and
parent material, proving to be promising tools for pedological studies, which
invariably demand for greater amounts of data to better comprehend pedogenetic
processes.