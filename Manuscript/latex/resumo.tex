Solos são essenciais para diversas atividades humanas; são a base para a
produção de alimentos, sustentação majoritária das obras de construção civil e
berço da maior parte da vegetação na crosta continental. Entretanto, o solo é um
corpo extremamente heterogêneo, variando não apenas ao longo da paisagem, mas
também em profundidade. Variação essa que possui implicações diretas em seu uso.
Até o presente, o estudo do solo usa principalmente artifícios mecânicos e
visuais para sua caracterização em campo, sendo auxiliado por análises
laboratoriais posteriormente. O objetivo deste trabalho foi testar a utilização
de novas tecnologias –sensores próximos– para aprimorar os métodos de
caracterização de perfis de solos e relações com sua gênese. Foram estudados
três perfis de solo com horizontes genéticos enterrados; i.e., solos
enterrados, formados pela deposição de sedimentos acima de solo preexistente. Os
perfis, localizados em Lavras, Minas Gerais, Brasil, foram classificados como
Organossolo Háplico (OX), Cambissolo Háplico (CX) e Gleissolo Háplico (GX). Os
perfis de solos foram analisados em campo e amostrados em um grid de 8 linhas
e 5 colunas (15 x 15 cm), perfazendo 40 amostras por perfil. Além de análises
químicas e físicas, as amostras foram analisadas por espectrometria de
fluorescência de raios-X portátil (pXRF) e espectroscopia de refletância difusa
na faixa de luz visível e infravermelho próximo (Vis-NIR). Os dados de pXRF
foram espacializados ao longo dos perfis para avaliar-se a distribuição espacial
dos teores elementares dentro e entre horizontes. Os dados espectrais do Vis-NIR
identificaram diferentes características dentro do horizonte hístico de OX em
função da profundidade, assim como as variações do teor de matéria orgânica no
horizonte A enterrado de CX. Na distribuição de teores elementares fornecida
pelo pXRF foram observadas descontinuidades na composição química dos três
perfis próximo a 110 cm de profundidade; indicação de diferenças químicas no
material de origem. Ambos os sensores detectaram maior teor de óxidos de ferro
no manto superficial do OX e GX, sugerindo que os sedimentos depositados têm sua
origem em Latossolos ou Argissolos desenvolvidos de granito-gnaisse e gabro,
classes de solo e materiais de origem comuns no entorno dos perfis estudados. Os
três perfis de solos apresentaram deposição de sedimentos provenientes das áreas
mais altas ao seu redor e, portanto, descontinuidade de seus materiais de origem
ao longo do perfil, evidenciada por contrastantes atributos químicos e físicos
expressos entre horizontes geneticamente distintos. Os sensores foram capazes de
detectar e quantificar características importantes que diferenciam os horizontes
entre si, além de variações dentro dos próprios horizontes, e sua gênese e
material de origem, provando-se ferramentas promissoras para estudos
pedológicos, nos quais há sempre grande demanda por maior quantidade de dados
para auxiliar a compreensão da pedogênese.